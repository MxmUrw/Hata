

% %\pagenumbering{alph}\setcounter{page}{1}%
% %\pagestyle{empty}
% \newgeometry{margin=0.6in}

% \begin{titlepage}
% %
% %\begin{center}
% \vspace*{1em}

% % minipage mit (Blind-)Text
% \begin{center}% {0.6\textwidth}
% \large {%
% 	TECHNISCHE UNIVERSITÄT BERLIN\\ 
%   \vspace{0.5cm}
%   \textbf{Fakultät IV: Elektrotechnik und Informatik}\\
% 	\textbf{Institut für Softwaretechnik und Theoretische Informatik}\\
% 	\textbf{Fachgebiet Modelle und Theorie verteilter Systeme}\\
% }
% \end{center}
% \hfill
% % \begin{minipage}{0.399\textwidth}
% % % \vspace*{-15px}
% % \vspace*{-15bp}
% % \parbox{\textwidth}{\includegraphics[height=2cm]{uni.png}}
% % \end{minipage}

% %\end{center}
% %
% \vspace*{0.07\textheight}
% %
% \begin{center}
% \huge { \textbf{Verified Type Checking} }\\
% % \vspace{0.8cm}
% % \Large{\textbf{A Formalization of Hindley-Milner Type Inference \\ in Agda}}
% \end{center}
% %
% \vspace*{0.07\textheight}
% %
% \begin{center}
% 	\LARGE { \textbf{Masterarbeit} }\\
% 	\vspace*{1em}
% 	\Large im Studiengang Computer Science \\
%   Ablieferungstermin: 16.\ Dezember 2021
% \end{center}
% %
% \begin{center}
% \vspace*{0.5em}
% \vspace*{0.9em}
% \end{center}
% %
% \begin{center}
% \Large {\textbf{Maxim Urschumzew}} \\
% \end{center}
% %
% \vspace*{0.05\textheight}

% % \begin{otherlanguage}{ngerman}
% \begin{center}
%   \Large
%   1. Gutachter: Prof. Dr.-Ing. Uwe Nestmann \\
%   2. Gutachter: Prof. Dr. Thorsten Altenkirch
% \end{center}
% % \end{otherlanguage}
% %
% \end{titlepage}

% \restoregeometry

% \cleardoublepage



\maketitle
%%%%%%%%%%%%%%%%%%%%%%%%%%%%%%%%%%%%%%%%%%%%%%%%%%%%%%%%%%%%%%%%%%%%

% \let\raggedsection\centering
% \section*{\abstractname}
% We present a formalization of Hindley-Milner type inference in Agda.
% This includes a unification algorithm. For the statement and solution
% of both problems the language of category theory is employed, and
% the features of such an approach are discussed.

% \begin{otherlanguage}{ngerman} 
%   \section*{\abstractname}
%   Wir stellen eine Formalisierung der Hindley-Milner Typinferenz in Agda vor.
%   Dies schließt die Definition und den Korrektheitsbeweis eines
%   Unifikationsalgorithmus ein. Zur Lösung beide Probleme wird
%   die Sprache der Kategorientheorie verwendet, und die Vor- und
%   Nachteile solch eines Ansatzes diskutiert.
% \end{otherlanguage}

% \let\raggedsection\raggedright

% \selectlanguage{english}
% \begin{abstract}
%   Abstract in English
% \end{abstract}

% \begin{abstract}
%   Abstract in German
% \end{abstract}


