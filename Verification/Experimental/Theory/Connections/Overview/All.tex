% Created 2021-06-19 Sat 17:25
% Intended LaTeX compiler: pdflatex
\documentclass[11pt]{article}
\usepackage[utf8]{inputenc}
\usepackage[T1]{fontenc}
\usepackage{graphicx}
\usepackage{grffile}
\usepackage{longtable}
\usepackage{wrapfig}
\usepackage{rotating}
\usepackage[normalem]{ulem}
\usepackage{amsmath}
\usepackage{textcomp}
\usepackage{amssymb}
\usepackage{capt-of}
\usepackage{hyperref}
\usepackage{quiver}
\usepackage{tikz-cd}
\usepackage{tikz}
\date{\today}
\title{}
\hypersetup{
 pdfauthor={},
 pdftitle={},
 pdfkeywords={},
 pdfsubject={},
 pdfcreator={Emacs 27.2 (Org mode 9.4.5)}, 
 pdflang={English}}
\begin{document}

\tableofcontents

\section{Theories}
\label{sec:orga630965}

The purpose of this \emph{theory of theories} is to explicitly describe a framework in
which all efforts in the area of syntax and semantics of type theories and programming languages
can be located. We want to include everything one would intuitively expect to be part of such research.
Thus we necessarily end up with a very rudimentary picture, based on a least common denominator.
Such a picture does not help with proving new theorems, and will not be surprising to anyone with familiar
with the usual concepts and goals. In fact it should merely reflect this state of mind in a formal environment.




\begin{tikzcd}
	&&&& {\mathbf{1}} \\
	\\
	&&&& {\text{TriniT}} && {} \\
	{\textbf{TypeT}_\le^{\text{infer}}} & {\textbf{TypeT}^{\text{check}}} \\
	&&& {\text{TypeT}} & {\text{CompT}} & {\text{CatT}} \\
	&&&& {\text{Theo}}
	\arrow[from=4-1, to=4-2]
	\arrow[from=5-5, to=6-5]
	\arrow[from=5-6, to=6-5]
	\arrow[dashed, from=3-5, to=5-6]
	\arrow[dashed, from=3-5, to=5-5]
	\arrow[from=5-4, to=6-5]
	\arrow[dashed, from=3-5, to=5-4]
	\arrow["\Lambda"{description}, curve={height=6pt}, from=1-5, to=3-5]
	\arrow["\Pi"{description}, curve={height=-6pt}, from=1-5, to=3-5]
	\arrow["\multimap"{description}, curve={height=18pt}, from=1-5, to=3-5]
	\arrow["\square"{description}, curve={height=-18pt}, from=1-5, to=3-5]
	\arrow["\forall"{description}, curve={height=24pt}, from=1-5, to=5-4]
\end{tikzcd}
\end{document}